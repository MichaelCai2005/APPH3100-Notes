\chapter{Quantum Systems}
\lecture{1}{Jan 24 01:10}{}
\section{Quantum Computing Systems}
\begin{eg}
Atoms: There is a nucleaus denoted \(Ze\) with an electron denoted as 
\(e^{-}\). Having a positively charged nucleaus and electron can
be described by a hamiltonian denoted as 
\[
  H^{at} = \frac{p^2}{2m_e} + V(r), V(r) = - \frac{1}{4\pi \epsilon_0} \frac{Ze^{2}}{r}
\]  
Where the hamiltonian represented the movement of the electron.

In this potential, we get discrete energy states. More precisely,
we can solve the schrodinger equation denoted 
\[
  H^{at}  \phi (\vec{r}) = E_n \phi (\vec{r})
\]

The eigen energies have the form
\[
  E_n = -\frac{1}{2}m_e c^2 \frac{\alpha^{2}}{n^{2} }
\]
This has discrete energy values and the smallest energy is for n=1 Where
\(\alpha = \frac{e^2}{4\pi \epsilon_0 h c} \approx {\frac{1}{137} }\) 

The eigenfunctions have the form of the following:

\[
  \phi(\vec{r}) = R_{ne}(r) Y_{em}(\theta ,\phi )
\]
which includes the spherical harmonics. These eigenfunctions are 
also orthonormal wave functions. There are also the three quantum numbers
denoted as: h,l, and m which are the principal quantum number,
angular momentum, and projection angular momentum. 

Intuitively for the hydrogen model, we see that the energystates
come close and close as a factor of \( \frac{1}{n^{2} }\)
\begin{remark}
  Simplification: We only use a two level system denoted as 
  \(E_1, E_2\) with wavfunction denoted as \(\psi_{100}, \psi_{210} \) 
  which are going to be known as the s-orbital and the p-orbital which is a 
  two level system.  
\end{remark}
\end{eg}

\begin{eg}
    Spin \(-\frac{1}{2}\): These systems include electrons, protons, and neutrons
    which have spins which are an intrinsic form of angular momentum. All these
    are spin \(\frac{-1}{2}  \)  particles. Spin always comes with a magnetic moment. 
    One example is a electron and it has two states which are denoted as 
    \[
      \ket{\uparrow}, \ket{\downarrow}
    \]
    These are orthoganal to each other with angular momentums \(\frac{h}{2} \text{and} -\frac{h}{2}\). 
    These electrons are in a magnetic field pointing in the z direction \(B = B \hat{e_Z}\)
    
    The interaction is denoted as 
    \[
      \hat{H_z} = \vec{\mu} \cdot \vec{B} = - (-g_s \mu_B \vec{S}/h)\cdot B
    \]
    where \(\mu_B = \frac{eh}{2m_e}\) which is the Bohr magneton. \(g_s =2  \)   
    which is known as the g-factor. The spin operator can be denoted in the following way
    \[
      \vec{S} = ( S_x, S-y, S_z)^T
    \]
    which is simply \(\frac{h}{2} \)  multipled with the Pauli matrices. We can rewrite this as 
    \[
      g_s \mu_B \frac{S_z}{\overline{h} } B_z = \mu_B B_z \sigma_z
    \]
    Thus we get that 
    \[
      H_z \ket{\uparrow} = \mu B_z \ket{\uparrow}
    \]
    and negative for the latter.

    With a 0 magnetic field, the spin up and spin down are the same. When you increase the energy,
    they have different states denoted as \(\mu_B B_Z\) and \(-\mu_B B_z\). 
    What you get here is called level splitting
\end{eg}

\begin{eg}
Photon Polarization: the electric field of a light wave in the z-direction may be 
in the form where the wave propagates in the z-direction. This is going to polarized
in the plane of the chalk board. This can be written as 
\[
  \overline{E} (z,y) = (E_x, E_y e^{i \theta }) e^{i(kz-\omega t)}
\]

where the values are known to be the angular frequency, wavelength, and wave vector.
We can rewrite this polarization vector as 
\[
  E_x \hat{e_x} + E_y e^{i \theta } e_y 
\]
with \( E_0 = \sqrt{E_x ^{2}  + E_y ^{2} } \). Thus it could have both an x and y component. 
If we have that \(\theta  = 0 , E_y = 0 \implies E_x = E_0\)  which is 
denoted as vertical linear polarization and the latter is called horizontal linear 
polarization if \( E_y = E_0\). Each photon has its own unique polarization! 

Polarization is a property carried by the individual photons that constitute the light field. 
\[
  \vec{V} = E_0 \hat{e_x}, \vec{H} = E_0 \hat{e_y}
\]
The polarization state vector of a single photon can be written as a superposition of 
\( \ket{H} + e^{-i \phi } \ket{V}\) which is a two level system 
\end{eg}

\begin{eg}
Superconducting Qubits: Suppose we have a LC-circuit with a inductor and a capacitor which
creates a resonance with a frequency with angular frequency \(\omega_c = \frac{1}{\sqrt{2c} }\)

Suppose we super cool it down such that the wire becomes superconducting
and for the case where \(k_b T \ll  \overline{h} \omega_c\) is much smaller than the excitation frequency of the 
circuit, we get discrete energy states for the oscillation that correspond to a 
quantum harmonic oscillator. We can write the hamiltonian of the circuit in the form

\[
  \hat{H}  = \hat{q}^{2}/{2C} + \phi ^{2} /{2C}
\]
which is similar to the momentum and position operator we see in a quantum harmonic oscillator. 
\(\hat{\phi }\) is the magnetic flux operator, \(\hat{q}\) is the charge operator. 

This creates a quadratic potential well with energy states \(E_0 = h \frac{\omega}{2}\) with
spaces of \(h \omega \) . The energies thus can be expressed as 
\[
  E_n = h \omega_c (n + \frac{1}{2})
\]
If you want this to be a two level system, it is practically bad since you might couple to many other states due to
its equal spacings. 
We can replace the inductor \(l\) with a Josephson junction (quantum tunneling/nonlinear inductor)
denoted as \(L_g\) . The hamiltonian of this system is not modified as 
\[
  \hat{H} = \frac{q^2}{2C} -E_y \cos \phi 
\]
which is a anharmonic oscillator with energy levels that are different with 
spacings that get smaller and smaller as it goes up. This is useful for defining
a two level system practical for control purposes where the energy levels 
\(\omega _{21} < \omega _{10} \approx 2 \pi \cdot 5 GHz\) which can be controlled by microwaves. 
Thus we get a two level system!

\end{eg}

\begin{remark}
  Linear Algebra Review:

  \begin{definition}
    wave functions can be interpreted as vectors in a vector space \(V\)  which
    is a hilbert space. We denote these in Dirac notation by \(\ket{\psi}\) . 
    For linear combinations. For \(c_i \in C \text{and} \ket{\psi_i} \in V\)  we have that 
    \[
      \ket{\psi } = c_1 \ket{\psi_1} + \dots + c_n \ket{\psi_n} \in V
    \]

    A set fo vectors \(\ket{\psi}\)  is linearly independent if none of the vectors can be written
    as a linear combination of the other vectors. In additional, if such a set spans the entire vector 
    space, it is called a basis. 
  \end{definition}

  \begin{definition}
    Two level systems are described by 2 dimensional vector spaces in \(C^2\) 
    spanned by two basis vectors \( \ket{v_1} \ket{v_2}\) . Any vector in that 
    vector space can be expressed as 
    \[
      \ket{\psi} = c_1 \ket{v_1} + c_2 \ket{v_2}
    \]
    Column vectors can be written as 
    \[
      \ket{\psi} = (c_1 c_2)^T
    \]
    and the complex conjugate can be written as 
    \[
      \ket{\psi} = (c_1 c_2)^{*}
    \]
    The scalar product is denoted as 
    \[
      \bra{\beta }\ket{\alpha } = (b_1^* b_2^*) (a_1 a_2)^T
    \]
  \end{definition}
\end{remark}