\lecture{5}{Jan 24 01:10}{}
\section*{1.\ Infinite Square Well --- Stationary States}

Consider a 1D infinite square well of width $a$, running from $x=0$ to $x=a$, with 
\[
V(x) \;=\; 
\begin{cases}
0, & 0 < x < a,\\
\infty, & \text{otherwise}.
\end{cases}
\]
The stationary‐state wavefunctions are
\[
\psi_n(x) 
\;=\; 
\sqrt{\frac{2}{a}}\,
\sin\Bigl(\frac{n \pi x}{a}\Bigr),
\quad
n = 1,2,3,\dots
\]
and the corresponding energies are 
\[
E_n 
\;=\;
\frac{\hbar^2 \pi^2}{2 m a^2}\,n^2.
\]
These eigenstates form an \emph{orthonormal} basis in the sense that
\[
\int_{0}^{a}\psi_m^*(x)\,\psi_n(x)\,dx 
\;=\;\delta_{m,n}.
\]
Any wavefunction $\psi(x)$ satisfying the boundary conditions can be expanded as
\[
\psi(x) \;=\; \sum_{n=1}^{\infty} c_n\,\psi_n(x).
\]

\section*{2.\ Complex Vector Space and Inner Products}

\paragraph{Finite‐Dimensional Analogy.}
We can think of a general complex vector 
\(\vec{r} \in \mathbb{C}^N\) as
\[
\vec{r} 
\;=\;
\begin{bmatrix}
c_1\\ 
c_2\\
\vdots\\
c_N
\end{bmatrix},
\]
with an inner product 
\[
\vec{r}^{\,\dagger}\,\vec{r} 
\;=\; 
\sum_{i=1}^N c_i^*\,c_i 
\;=\; 
\|\vec{r}\|^2.
\]
The \emph{outer product} $\vec{r}\,\vec{r}^\dagger$ is a matrix (an operator) formed by taking all pairwise products of components.

\paragraph{Ket and Bra Notation (Dirac).}
In quantum mechanics, we write a vector as a \emph{ket}:
\[
\ket{\alpha}
\;=\;
\sum_{i} a_i \,\ket{u_i},
\]
where $\{\ket{u_i}\}$ is a basis of the Hilbert space and $a_i \in \mathbb{C}$ are coefficients.  The \emph{bra} $\bra{\alpha}$ is the Hermitian‐conjugate (row‐vector) version:
\[
\bra{\alpha} 
\;=\;
\bigl(\ket{\alpha}\bigr)^\dagger
\;=\;
\sum_{i} a_i^*\,\bra{u_i}.
\]
The inner product $\braket{\beta}{\alpha}$ is a scalar, generally complex:
\[
\braket{\beta}{\alpha} 
\;=\;
\sum_{ij} b_i^*\, a_j\,\braket{u_i}{u_j}
\;=\;
\sum_i b_i^* \, a_i
\quad
(\text{if }\ket{u_i}\text{ are orthonormal}). 
\]
For orthonormal $\{\ket{u_i}\}$, we have $\braket{u_i}{u_j} = \delta_{ij}$, and the completeness relation
\[
\sum_{i}\ket{u_i}\bra{u_i} 
\;=\; 
\mathbf{1}.
\]

\section*{3.\ Operators}

Operators $\hat{O}$, $\hat{H}$, $\hat{p}$, etc.\ act on kets to produce new kets:
\[
\hat{O}\,\ket{\alpha} 
\;=\;
\ket{\beta}.
\]
The bra version is obtained by taking the Hermitian conjugate:
\[
\bra{\alpha}\,\hat{O}^\dagger 
\;=\;
\bra{\beta}.
\]
For a Hermitian operator (an observable) $\hat{O} = \hat{O}^\dagger$. 

\paragraph{Examples.}
- $\hat{H}\ket{\psi_n} = E_n \ket{\psi_n}$, the energy‐eigenvalue equation.  
- Linearity: \(\hat{O}(\ket{\alpha}+\ket{\beta}) = \hat{O}\ket{\alpha} + \hat{O}\ket{\beta}\).  

\paragraph{Noncommuting Operators.}
The commutator of two operators $\hat{O}_1, \hat{O}_2$ is
\[
[\hat{O}_1, \hat{O}_2]
\;=\;
\hat{O}_1\,\hat{O}_2 
\;-\; 
\hat{O}_2\,\hat{O}_1.
\]
For position and momentum in 1D, we have 
\(
\bigl[\hat{p}, \hat{x}\bigr] = -i\hbar,
\)
so $\hat{p}\,\hat{x} \neq \hat{x}\,\hat{p}$.  In general, if two operators \emph{do} commute, they can be \emph{simultaneously diagonalized}.  If they \emph{do not} commute, measuring one observable disturbs the other.

\section*{4.\ Outer Products as Projectors}

The outer product \(\ket{\alpha}\bra{\alpha}\) is itself an operator.  For instance, acting on a state \(\ket{\beta}\) gives
\[
(\ket{\alpha}\bra{\alpha})\,\ket{\beta}
\;=\;
\ket{\alpha}\,\bigl(\braket{\alpha}{\beta}\bigr),
\]
which is a vector parallel to $\ket{\alpha}$ (scaled by the inner product).  If $\ket{\alpha}$ is normalized, \(\ket{\alpha}\bra{\alpha}\) \emph{projects} any vector onto the direction of $\ket{\alpha}$.

\paragraph{Completeness (Resolution of the Identity).}
If $\{\ket{u_i}\}$ is an orthonormal basis, then
\[
\sum_{i}\ket{u_i}\bra{u_i} 
\;=\;
\mathbf{1},
\]
where $\mathbf{1}$ is the identity operator.  This is a powerful tool for expanding or inserting identities in quantum‐mechanical derivations.
